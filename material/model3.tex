\subsubsection{Fundamentos biológicos}

La idea de que el sistema inmunitario interviene en el desarrollo del cáncer, mediante  respuestas inmunitarias innatas y adaptativas para eliminar los patógenos microbianos y las células cancerosas aberrantes mediante la exploración del organismo no es nueva \cite{Burkholder2014}. De hecho, es, precisamente, un fallo en la vigilancia de las células tumorales por parte del sistema inmunitario o una insuficiencia en la supresión inmunitaria de las células tumorales, el desencadenante del crecimiento tumoral y la invasión tisular cancerígena \cite{Weinberg1996, Kleinsmith2001}.\\

Las células inmunitarias innatas (dendríticas y natural killer) y las adaptativas (linfocitos T y B) intervienen en el microentorno tumoral, comunicándose entre sí mediante la producción de citocinas y quimiocinas como método de señalización y reclutación \cite{Kuznetsov1994, dePillis2005, Chew2012}. Aunque son muchas las células inmunitarias que intervienen en el tumor (macrófagos, monocitos, Natural Killer, linfocitos T y B, etc.), las células inmunitarias más activas contra el tumor son las Natural Killer (NK) junto a los Macrófagos Asociados a Tumores (TMA) por parte del sistema innato y los Linfocitos T Citotóxicos (CTL) por parte del sistema adaptativo \cite{Grivennikov2010}. El sistema inmunitario es, por tanto, un elemento fundamental en la localización y acción contra el cáncer, por lo que su integración dentro del modelo resulta indispensable.\\

En el proceso de acción inmunitaria intervienen varios procesos (o fases) que pueden resumirse en: búsqueda y localización de células tumorales, respuesta inmuno-innata (reconocimiento y acción inmune general), activación de células T, generación y viaje de CTL al tumor, respuesta inmuno-adaptativa (infiltración y acción específica antitumoral) \cite{Charles2001}.

\subsubsection{Modelización del sistema inmune en MSCAM}

Para modelizar el sistema inmune dentro del modelo MSCAM, es necesario añadir al modelo previo de crecimiento tumoral avascular ciertos elementos que describiremos a continuación, y que comprenden nuevos estados, variables y reglas de transición. Más concretamente:\\

\begin{itemize}
    \item Se contemplan 4 nuevos estados para las celdas: 
 
  \begin{table}[H]
	\centering 
    \renewcommand{\arraystretch}{1.2}
         \begin{tabular}{|c|m{5cm}|c|}  \hline
 			4 & Célula Natural Killer & NK\\ \hline
 		    5 & Linfocito T Citotóxico & CTL\\ \hline
		\end{tabular} \,
		\begin{tabular}{|c|m{5cm}|c|}  \hline
			6 & Célula inestable & US \\ \hline
			7 & Célula en muerte programada & DC \\ \hline
		\end{tabular} 
	\caption{Estados de las celdas referentes al S.I.}
	\label{table:estadosSI}
\end{table}

Los estados \textit{NK} y \textit{CTL} representan a los agentes del sistema inmune innato y adaptativo respectivamente. Los estados \textit{US} y \textit{DC} corresponden a estados derivados de las acciones inmunes contra el tumor. En adelante, veremos las caracterizaciones y propiedades de cada uno de estos estados de forma detallada.\\

\item La presencia de antígenos como sustancia atractiva para los agentes inmunes del sistema innato, es modelada a través de un modelo de difusión, de forma que cada vez que una \textit{PC} prolifera, en las nuevas zonas de proliferación se deposita un pequeña cantidad $\varsigma_{PC}$ de antígeno que se difunde a sus vecinos (Moore). Esta señal de antígenos será enriquecida por las células inmunes cuando entren en contacto con células tumorales, secretando una pequeña cantidad $\varsigma_{I}$.\\

\item Dado que se considera que la proliferación tumoral se debe a la falta de acción del sistema inmune, en el modelo no se consideran celdas inmunes hasta que el tumor llega a cierto tamaño. A partir de entonces, se fija una pequeña proporción de celdas de células pertenecientes al sistema inmune innato, que se mantendrá como umbral, $\upsilon_{NK}$ durante todo el proceso. Más concretamente, en cada paso se inspeccionará cada celda \textit{HC} y será reemplazada por una \textit{NK} de acuerdo a la probabilidad: 
\begin{equation}
    P_{bNK} = \max \left\lbrace 0, \upsilon_{NK} - \frac{1}{\textit{ncell}} \left\vert \left\lbrace c \in \textit{lattice} \, \vert \, state(c) = \textit{NK} \right\rbrace \right\vert \right\rbrace 
\end{equation}

\info{V - El valor de $\upsilon_{NK}$ es habitualmente tomado en la literatura como 0.1} \info{V - En la implementación, en realidad conviene comprobar de antemano si $|\textit{c with state=NK}|$ es mayor que el umbral, de forma que si lo es no se añada ninguna}


\item Dichas células realizan caminos aleatorios guiadas probabilísticamente por la señal de antígenos. Cuando estas células entran en contacto con una célula tumoral, se dan dos procesos.  Por una parte, intentan destruirla. El destino de esta célula tumoral depende de la capacidad citotóxica intrínseca de las células inmunitarias así como de la resistencia adquirida por las células tumorales y el daño sufrido por estas, más concretamente la probabilidad de que una célula tumoral desaparezca a causa de una acción de una célula inmune corresponde a: 

\begin{equation}
    P_{lys}(c_t, c_i) = \min \left\lbrace 1, 2 - \exp{\left(0.69 \cdot (\mathcal{R}(c_t) - \mathcal{D}(c_t) - \mathcal{T}(c_i)) \right)} \right\rbrace
\end{equation}

$\mathcal{R}$ representa la resistencia de las células tumorales ante la acción de las células inmunes. Este parámetro nos permite modelar la resistencia intrínseca de las células así como la adquirida por las propias células tumorales. $\mathcal{D}$ representa el daño sufrido por una célula tumoral a causa de la acción inmune y $\mathcal{T}$ la capacidad citotóxica de las distintas células inmunes. Este último, nos permite representar la diferente efectividad de las células innatas (\textit{NK}) y las adaptativas (\textit{CTL}) así como los microentornos tumorales inmunodeprimidos.\\

\item Cuando una célula tumoral muere a causa de la acción inmune, dicha célula pasa a un estado inestable, y muere tras un cierto número de pasos, pudiéndose regenerar (reparación tisular) si posee a su alrededor 5 o más células sanas.\\

\item En caso contrario (la célula tumoral resiste), entonces sufre cierto daño debido al conflicto \textit{IC vs TC}, pero también adquiere cierto incremento en su resistencia al sistema inmune. La resistencia se propaga en el proceso de proliferación tumoral, mientras que el daño es restablecido a $0$ cada vez que la célula es reemplazada en el proceso de proliferación.\\  


\item Como parte del proceso de reclutación, cuando una célula inmune (\textit{NK} o \textit{CTL}) destruye una célula tumoral, se explora cada celda no tumoral que la rodea, que pasa a ser una célula inmune del mismo tipo con probabilidad: 

\begin{equation}
    P_{rec} = \exp{\left(-\frac{1}{\xi^2} \cdot \left( \sum_{i \in \eta_1} T_i  \right)^{-2} \right)}
\end{equation}\\

Donde $\xi$ representa el factor de reclutamiento. De manera que, cuanto mayor sea el factor de reclutamiento y el número de células cancerosas a su alrededor, mayor será el número de células reclutadas. Nótese además que la reclutación de una \textit{CTL} puede sustituir a una $NK$ pero no al revés.\\

\item Adicionalmente, en el proceso de respuesta del sistema innato y la activación del sistema inmune, tras cualquier conflicto \textit{TC vs NK} la célula $NK$ desaparece, cambiando de estado a una celda $CTL$ que permanecerá inactiva durante cierto tiempo hasta que el sistema adaptativo se considere activo. \unsure{V - Aquí he de mirar la bibliografía a ver cuánto tarda en comenzar la respuesta inmune, para poder determinar el número de pasos (en base a las 5h por paso).}\\

\item A diferencia de las $NK$, las células $CTL$ pueden ejecutar varios ataques (secuenciales) contra las células tumorales. Además al igual, que las células $NK$, los agentes inmunológicos mueren al cabo del tiempo. La probabilidad de que una célula inmune desaparezca, dando lugar a una célula sana, viene dada por:

\begin{equation}
    P_{dI} = \exp{\left(- \frac{1}{\lambda_I \cdot \textit{age} + \textit{fights}^{\sigma}}\right)}
\end{equation}\\

\info{V: valores de $\lambda_I=0.18$ y $\sigma=0.8$ pueden ser valores adecuados}

Donde \textit{age} representa el número de pasos que lleva activa la célula inmune y $\lambda_I$ el factor de longevidad de las células inmunes, \textit{fights} representa el número de conflictos en los que ha participado la célula y $\sigma$ el factor de regulación de la supervivencia (sólo para los \textit{CTL}). 

\end{itemize}






