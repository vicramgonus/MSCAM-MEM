Los cánceres son enfermedades muy complejas y diversas, impulsadas por mutaciones genéticas, aberraciones del número de copias, y epigenéticas, que alteran una multitud de vías de señalización que cambian los fenotipos, como la proliferación, la supervivencia o la muerte de las células, la reparación, movimiento e invasión, etc. Un cáncer no puede considerarse un organismo estático, sino más bien, una especie en evolución (expansiva y adaptativa). El desarrollo del cáncer no es estable, sino que está influenciado por diversos factores tanto entre distintos tipos de cáncer, dentro del mismo tipo en diferentes individuos, e incluso dentro de distintas poblaciones de células tumorales de un mismo tumor \cite{Gerlinger2014, Tabassum2015}. Además, al igual que otros agentes biológicos, los mecanismos de adaptación de las células cancerosas pueden dar lugar a resistencia a los fármacos, ya que el tumor reacciona redirigiendo las señales para restaurar la homeostasis en respuesta a los fármacos dirigidos \cite{Hartman2001, Morken2014, Echeverria2019}. Por todo ello,  el cáncer ha de entenderse como un sistema holístico, dinámico, reactivo y en evolución.\\

A pesar del enorme éxito de la biología molecular del siglo XXI, en la mayoría de los casos no existe una explicación exacta y coherente de los mecanismos subyacentes del cáncer. Aunque los enfoques genómicos, transcriptómicos y proteómicos contribuyen a nuestro conocimiento profundo del cáncer, no pueden traducirse directamente en mejores tratamientos. Debemos comprender cómo los factores genómicos, transcriptómicos, proteómicos, de señalización celular y microambientales interactúan como sistema para influir en el desarrollo, la progresión y la resistencia al tratamiento del cáncer. Aunque la tarea de estudio clínico ha recaído tradicionalmente en estudios in vivo, los resultados experimentales obtenidos pueden verse muy atenuados en la implantación clínica de los métodos propuestos, debido en muchos casos a realizar de experimentos a nivel de todo el sistema sin utilizar sistemas inferiores (por ejemplo, ratón frente a humano), comprometer el sistema (por ejemplo, eliminando genes) o ignorar características clave del sistema (por ejemplo, la respuesta del sistema inmunitario). Son necesarios métodos más inclusivos que integren las características celulares, genómicas, microambientales y espaciales de los cánceres para poder entender y superar sus multifacéticos mecanismos de resistencia.\\ 

A pesar de la complejidad innata de los sistemas biológicos, existen puntos comunes y principios subyacentes que rigen su comportamiento \cite{Hartwell1999}, ya que dichos sistemas, y especialmente los tumores cancerosos, dan lugar a fenotipos reproducibles y son capaces de tomar decisiones, no necesariamente deterministas pero sí identificables \cite{Hartman2001, Siegal2002}. Al descubrir y codificar los elementos comunes, los principios e interacciones de los sistemas biológicos como modelos computacionales podemos esperar dilucidar la función biológica y predecir el efecto de las perturbaciones, como las mutaciones genéticas o los fármacos, en tipos de cáncer específicos para ofrecer tratamientos más precisos y eficaces de las enfermedades \cite{DiVentura2006}. Aunque un modelo computacional no puede capturar todas las características observadas en un sistema real, sí hace explícito lo que puede y no puede incluirse, dejando claro, a priori, posibles limitaciones y sesgos del mismo, permitiendo establecer una caracterización de sus puntos fuertes y débiles. Los modelos computacionales proporcionan un medio para representar matemáticamente las explicaciones mecánicas de los comportamientos biológicos a nivel genético, molecular y celular, y para especificar cómo evolucionan e interactúan estos comportamientos. En esencia, estos modelos tratan los sistemas biológicos como redes de agentes biológicos programados que interactúan entre sí, ofreciendo una visión mecanicista de todo el sistema. Características como la abstracción, modularidad, composicionalidad y concurrencia permiten el desarrollo de modelos multiescala de sistemas biológicos complejos, permitiendo establecer ciertos razonamientos sobre el comportamiento dinámico, las propiedades y las respuestas de los sistemas biológicos \textit{in silico} como complemento a los enfoques experimentales clásicos. En las dos últimas décadas, este enfoque de ha dado lugar a la cooperación entre los estudios biológicos y los enfoques computacionales, como la modelización matemática, la bioinformática y el aprendizaje automático. En esta perspectiva hacemos una breve revisión de distintos enfoques de modelización computacional del cáncer, los resultados alcanzados y los retos y perspectivas futuras.\\

Los modelos matemáticos propuestos para el crecimiento de los tumores pueden clasificarse en tres categorías generales: modelos continuos, discretos e híbridos. Los modelos continuos describen el fenómeno del crecimiento mediante ecuaciones diferenciales ordinarias (EDO) o ecuaciones en derivadas parciales (EDP). Los ODE pueden utilizarse para estudiar el crecimiento de las poblaciones de células tumorales. Sin embargo, las estructuras espaciales no pueden ser captadas por las EDO. Los modelos que utilizan las EDP pueden expresar mejor las propiedades temporales y espaciales del crecimiento tumoral. Sin embargo, la solución numérica de las EDP, especialmente en presencia de términos de difusión, es un problema crucial en sí mismo.\\

Desde los primeros modelos sencillos de crecimiento de tumores sólidos \cite{Burton1966}, basándose en la limitación de la concentración de nutrientes, se han publicado numerosos modelos computacionales \cite{Anderson1998, Bellouquid2013, Smolle1993, Sabzpoushan2018, Robertson2015} examinando diferentes características del cáncer, como la angiogénesis, la invasión tumoral, la interacción entre las células inmunitarias y las células cancerosas, así como el efecto de los factores microambientales en el crecimiento del tumor. Otros \cite{Byrne2006, Materi2007, Kuznetsov, Eikenberry2009, Eftimie2011, Rejniak2011, Marcu2012} han incorporado a los aspectos previos, la interacción del sistema inmunitario con las células tumorales. Asimismo, el efecto de diferentes fármacos y concentraciones sobre el desarrollo tumoral ha sido modelado en varios estudios \cite{Aroesty1973, Simeoni2004, Pillis2007, Pillis2008}, haciendo uso de diversas técnicas y centrándose en distintos ámbitos. Algunos factores muy relevantes  en el desarrollo tumoral, y estudiados mediante modelos computacionales, corresponden al factor de resistencia a los fármacos \cite{Marcu2004}, la repoblación tumoral \cite{Marcu2010} y el proceso de angiogénesis \cite{Wang2013}.\\

Dado el carácter discreto tanto de la evolución tumoral como de la aplicación de los regímenes de quimioterapia, parece que el uso de un modelo discreto puede reflejar de forma más realista la dinámica del crecimiento del tumor bajo el efecto de la terapia. Un modelo discreto nos permite simular directamente situaciones complejas con sólo pequeños cambios en los parámetros fisiológicos del modelo. Los autómatas celulares (AC) son un formalismo común para los modelos discretos, permitiéndonos además evitar el uso de complicadas ecuaciones diferenciales. Los AC pueden considerarse herramientas sencillas para modelar sistemas complejos autoorganizativos y emergentes, investigando cómo las relaciones simples entre componentes más pequeños dan lugar a los comportamientos colectivos de un sistema, permitiendo la modelización de comportamientos globales a partir de interacciones a nivel microscópico. Por lo tanto, la estructura macroscópica de un tumor puede reproducirse a partir de los procesos microscópicos utilizando el AC \cite{Mallet2006, Schmitz2002, Reis2009}. Además, parece que la modelización de las reglas biológicas con el formalismo CA es más natural de cara a la información experta y empírica de oncólogos e investigadores.\\

Además del carácter discreto comentado, también se han expuesto previamente algunas notas acerca del carácter no determinista y estocástico en el proceso de desarrollo tumoral, tanto en la proliferación de las células cancerosas como en la reacción inmunitaria y la acción farmacológica. Esto, junto con lo ya comentado con anterioridad, nos sugiere el modelo de Autómatas Celulares Estocásticos \cite{Dobrushin1978} como un modelo natural para el problema y que nos permite recoger de forma fiel las características intrínsecas del problema como características intrínsecas del modelo.\\

\textbf{Estructura del trabajo.} En las \hyperref[sec:model]{\textit{Sección 2}} se desarrolla una descripción detallada del modelo \textit{MSCAM}, estableciendo los elementos característicos del modelo (topología del espacio, descripción de estados y variables, reglas de transición estocásticas) y las bases biológicas que fundamentan el diseño. \hyperref[sec:impl]{\textit{Sección 3}} se detalla la implementación realizada bajo la tecnología \textit{NetLogo} y se comentan las adaptaciones técnicas realizadas como aproximaciones al modelo propuesto. \info{\textbf{V}: Se completará un poco más en detalle cuando el trabajo esté completo}


